\chapter{对抗训练中的正则化方法}

深度学习中对抗样本的存在性是一个反直觉的特性,一个被普遍接受的解释是:对抗样本是模型的内在的盲点区域,这个盲区的结构与数据分布紧密相关,以一种不那么明显的方式\cite{szegedy2013intriguing}。

这种不明显体现在,对抗样本和干净样本在我们人类看来区别并没有那么地明显,甚至不可察觉,但事实上,对于模型来说,这种不明显的区别是致命的。

这种致命体现在,两个在原始特征空间中非常邻近的样本点,经过深度网络一系列非线性变换后,两个样本点在隐空间中的距离竟然会变得很远,以至于最终落在了模型的两个不同的判决区域。

一个直觉的方案是在隐空间中尽量拉近对抗样本与干净样本的距离,期望使得模型不再将两个样本点(对抗样本和原始样本)判为不同类别。这个思想就是度量学习中的对比损失,学习一个良好的深度网络,使得在隐空间中对抗样本和干净样本之间的距离尽量小。

对比损失可以形式化为:
\begin{equation}
    L
    = 
    \frac{1}{(|\mathcal{D}| + |\mathcal{A}|)^2} \sum_{x_i, x_j \in \mathcal{D} \cup \mathcal{A}} w_{ij}\big\|\varphi(x_i) - \varphi(x_j)\big\|_2^2
    \ ,
    \label{contrastive_loss}
\end{equation}
其中$w_{ij}$定义为:
\begin{equation}
    w_{ij} = \left\{
    \begin{array}{rl}
    1  & \quad \textnormal{if } (x_i, x_j) \in \mathcal{P}_{sim} \ ,\\
    -1 & \quad \textnormal{if } (x_i, x_j) \in \mathcal{P}_{dis} \ ,\\
    0  & \quad otherwise \ .
    \end{array} \right.
    \label{metric_weight}
\end{equation}
其中$\mathcal{P}_{sim}$表示相似样本对集合,$\mathcal{P}_{dis}$表示不相似样本对集合。这里的式\eqref{contrastive_loss}在度量学习中只是一个最基本的形式,本文这样写是为了阐明本文思路,对比损失的其它变体形式在这里不予赘述。

为了将对比损失的思想作为一个正则化项,合适地融入对抗训练的损失中,下面介绍三种构造方式:对抗逻辑配对(Adversarial Logit Pairing, ALP)\cite{kannan2018adversarial}、结合域适应的对抗训练(Adversarial Training with Domain Adaption, ATDA)\cite{song2018improving}、对抗流形正则化(Adversarial Manifold Regularization, AMR)。

\section{ALP} \label{section:alp}

ALP可以定义为:
\begin{equation}
    L_{alp}
    = 
    \frac{1}{|\mathcal{D}|} \sum_{x_i \in \mathcal{D}} \big\|\varphi(x_i) - \varphi(x_i^{adv})\big\|_2^2 \ ,
    \label{alp}
\end{equation}
其中$x_i^{adv}$是$x_i$对应的使用FGSM攻击方法生成的对抗样本。式\eqref{alp}旨在缩小每个样本点和它对应的对抗样本点间的欧式距离,即在式\eqref{metric_weight}中,$\mathcal{P}_{sim} = \{(x_i, x_i^{adv}) | x_i \in \mathcal{D} \}$,$\mathcal{P}_{dis} = \varnothing$。

Kannan等人在提出ALP的文章中,他们仅仅在ImageNet上进行了有效性验证,在FGSM攻击的对抗训练过程中增加ALP正则化项,发现这样训练的模型对FGSM敌人和PGD敌人的鲁棒泛化能力都得到了提高\cite{kannan2018adversarial},随后却被人发现通过简单的增加PGD攻击的迭代次数便可将PGD鲁棒精度降为0.6\%\cite{engstrom2018evaluating},这篇文章因此从NIPS 2018撤稿。本文将在MNIST、SVHN等数据集上对ALP的有效性进行验证,并画出PGD鲁棒精度随迭代次数增多而下降的收敛曲线,以保证结论的准确性,详见章节\ref{section:convergence}。

\section{ADTA}

把对抗样本和干净样本看成两个不同的域,结合域适应的思想,ATDA提出用最大均值散度(Maximum Mean Discrepancy, MMD)\cite{borgwardt2006integrating}和相关性匹配(CORrelation ALignment, CORAL)\cite{sun2016deep}作为正则化项提高对抗训练的泛化能力\cite{song2018improving}。这和对比损失中点到点的拉近目标是有所区别的,MMD和CORAL是一种集合到集合的拉近。

MMD的目标是最小化干净样本和对抗样本的均值向量间的距离:
\begin{equation}
    L_{mmd} 
    = 
    \frac{1}{k} \Big\| \frac{1}{|\mathcal{D}|} \sum_{x\in\mathcal{D}}F(x_i) - \frac{1}{|\mathcal{A}|} \sum_{x^{adv}\in\mathcal{A}}F(x_i^{adv}) \Big\|_1 
    .
\end{equation}

CORAL的目标是最小化干净样本和对抗样本间的协方差矩阵间的距离:
\begin{equation}
    L_{coral} 
    = 
    \frac{1}{k^2} \Big\| C_{\varphi(\mathcal{D})} -  C_{\varphi(\mathcal{A})} \Big\|_{\ell_1} 
    ,
\end{equation}
其中$C_{\varphi(\mathcal{D})}$和$C_{\varphi(\mathcal{A})}$分别是干净样本和对抗样本在逻辑空间(Logit sapce)上的协方差矩阵,$\| \cdot \|_{\ell_1}$表示矩阵的$L_1$范数。

\section{AMR}

流形正则化(Manifold Regularization, MR)在早期被提出,主要是希望通过这种方式去利用大量的未标记样本,来提高分类或回归任务的性能。它的主要思想是利用数据集的几何性状去对模型空间产生约束。该正则化项可以被估计为\cite{belkin2006manifold}:
\begin{equation}
    L_{mr}
    = 
    \frac{1}{|\mathcal{D}|^2} \sum_{x_i, x_j \in \mathcal{D}} w_{ij}\big\|\varphi(x_i) - \varphi(x_j)\big\|_2^2
    \ ,
\end{equation}
其中$w_{ij}$定义为:
\begin{equation}
    w_{ij} = \left\{
    \begin{array}{rl}
    \exp \left(\frac{-\left\|x_i-x_j\right\|_2^2}{t}\right)  & \quad \textnormal{if } (x_i, x_j) \in \mathcal{P}_{k} \ ,\\
    0  & \quad otherwise \ .
    \end{array} \right.
\end{equation}
其中温度参数$t \in \mathbb{R}$,$\mathcal{P}_{k}$表示集合$\mathcal{D}$中所有样本间的k近邻构成的样本对集合。该正则化项的直觉是:在原空间中非常接近的样本对,经过变换后,尽量在隐空间中也相近;在原空间不那么靠近的样本对,在隐空间中的紧密度也就不那么重要了(体现在$w_{ij}$的相对大小中)。

本文将流形正则化适应到对抗训练的环境中,提出对抗流形正则化(Adversarial Manifold Regularization, AMR),其主要思想是在原空间中非常接近的干净样本和对抗样本(k近邻),在隐空间中也应该尽量靠近。AMR可以被形式化为:
\begin{equation}
\begin{aligned}
    L_{amr}
    = \ &
    \frac{1}{|\mathcal{D}| \cdot |\mathcal{A}|} \sum_{x_i^{cln} \in \mathcal{D}} \sum_{x_j^{adv} \in \mathcal{A}} w_{ij}^{(1)} \big\|\varphi(x_i^{cln}) - \varphi(x_j^{adv})\big\|_2^2 \\
    + & 
    \frac{1}{|\mathcal{D}| \cdot |\mathcal{A}|} \sum_{x_i^{adv} \in \mathcal{A}} \sum_{x_j^{cln} \in \mathcal{D}} w_{ij}^{(2)} \big\|\varphi(x_i^{adv}) - \varphi(x_j^{cln})\big\|_2^2
    \ ,
\end{aligned}
\end{equation}
其中$w_{ij}^{(l)}(l\in\{1, 2\})$定义为:
\begin{equation}
    w_{ij}^{(l)} = \left\{
    \begin{array}{rl}
    \exp \left(\frac{-\left\|\varPhi(x_i)-\varPhi(x_j)\right\|_2^2}{t}\right)  & \quad \textnormal{if } (x_i, x_j) \in \mathcal{P}_{k}^{(l)} \ ,\\
    0  & \quad otherwise \ .
    \end{array} \right.
\end{equation}
其中温度参数$t \in \mathbb{R}$,$\mathcal{P}_{k}^{(1)}$表示集合$\mathcal{D}$中所有样本在集合$\mathcal{A}$中的k近邻构成的样本对集合,$\mathcal{P}_{k}^{(2)}$表示集合$\mathcal{A}$中所有样本在集合$\mathcal{D}$中的k近邻构成的样本对集合。$\varPhi(\cdot)$是一个特征提取器(自编码器、自然训练的分类器或对抗训练的分类器),之所以使用一个特征提取器来计算k近邻的距离,是因为在高维的图像数据中欧式距离已失效。