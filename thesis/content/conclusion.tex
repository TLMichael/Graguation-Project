\chapter{总结与展望}

本文基于对抗训练,即用对抗样本并入深度神经网络的训练集,联合自然样本一起训练以增强神经网络的鲁棒性。为了进一步增加对抗训练的泛化能力,本文结合度量学习的概念,引申出了四种正则化项,其中对抗逻辑配对方法、最大均值散度方法和相关性匹配方法是已经被人应用到对抗训练中,而对抗流形正则化方法则是我们首次应用到对抗训练中。我们在实验部分探究了它们对模型的鲁棒泛化能力和损失敏感度的影响,结果现实正则化方法均有效地提升了神经网络的鲁棒泛化能力,并减小了损失敏感度,证明了我们方法的有效性。

未来的研究可以从以下几个方向入手:

(1)在对抗训练的语境下,探究四种正则化项之间的相互影响,探究多个正则化项的组合是不是可以进一步提升模型的鲁棒泛化能力。

(2)探究模型的深度和宽度等因素对对抗训练泛化能力的影响。

(3)探究多视图任务中对抗训练对模型鲁棒性的影响。

(4)探究对抗训练对迁移学习性能的影响。

(5)探究目标检测、语音识别等任务中对抗训练的实现方式。